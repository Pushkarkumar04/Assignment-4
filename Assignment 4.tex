\documentclass{article}
\usepackage[utf8]{inputenc}
\usepackage{graphicx}

\title{ASSIGNMENT 4\\ON\\Disruptive Innovations In Healthcare​}
\author{BY\\PUSHKAR KUMAR\\ROLL NO.: 21111040\\FIRST SEMESTER\\BRANCH:BIOMEDICAL ENGINEERING\\SECTION:A\\NATIONAL INSTITUTE OF TECHNOLOGY, RAIPUR\\ASSIGNMENT SUBMITTED TO\\
DEPARTMENT OF BIOMEDICAL ENGINEERING}
\date{}

\begin{document}

\maketitle
\begin{figure}[h]
    \centering
    \includegraphics[height=9cm,width=9cm]{download.jpg}
\end{figure}


\section{ Disruptive innovation in Healthcare}
Disruptive innovations are those that cause radical change and often result in new leaders in the field. They overturn the usual way of doing things to such an extent that they have a ripple effect throughout the industry.
\\Disruptive innovations in healthcare can influence a new system that provides a continuum of care focused on each individual patient's needs, rather than focusing primarily on complex disorders and urgent health crises.Examples of disruptive innovations in healthcare include the use of miniaturized blood glucose meters that patients can take along wherever they go. As a result, patients can self-manage most aspects of diabetes more effectively and conveniently, whereas in the past, they would have access to treatment solely through healthcare professionals.Here are the some most disruptive technologies in healthcare  
\subsection{Artificial Intelligence}
We’ve all seen sci-fi movies about an AI that threatens to overturn the human race and take over after developing a mind of its own, but that’s not quite the reality of it.AI applications can manage patient intake and scheduling as well as billing. Chatbots answer patient questions. With natural language processing capabilities, AI can collate and analyze survey responses. AI will probably increase in use as a way to bring down healthcare costs and let doctors and staff focus on patient care. Healthcare leaders must be knowledgeable about the issues surrounding database management and patient privacy. 
\subsection{Consumer devices, wearables, and apps     }
In the past, a patient could get only biometric data about their pulse, heart rate, blood oxygen, and blood pressure when they went to the doctor’s office. Now, consumers take charge of their own health journey, using data gathered from their Fitbits, smartwatches, and mobile phone fitness apps. Physicians can use the data gathered from these wearables to make treatment decisions, although the vast amount of personal information collected by these apps has led to legal and ethical concerns over data privacy.
\subsection{Blockchain         }
Blockchain is a database technology that uses encryption and other security measures to store data and link it in a way that enhances security and usability. This innovation facilitates many aspects of healthcare, including patient records, supply and distribution, and research. Tech startups have entered the healthcare sector with blockchain applications that have changed how providers use medical data. 
\subsection{Telemedicine     }
COVID-19 has undoubtedly accelerated the delivery of telemedicine, and experts affirm that telemedicine is here to stay. It’s effective, doctors will be reimbursed for a telehealth consultation, and many patients prefer it. However, telemedicine is highly dependent on internet access, and some areas of the U.S. still have poor connectivity.
\subsection{3D Printing}
3D-printing can bring wonders in all aspects of healthcare. We can now print
biotissues, artificial limbs, pills, blood vessels and the list goes on and will likely keep on doing so.Burn victims are using 3D printed skin and some doctors are even using their patient’s plasma and skin biopsies to create new human skin – instead of using skin grafts as they can be painful and not as visually appealing. From printing novelty objects to hearing aids to prosthetic limbs and all the way to spacecraft engines, 3D printing technology is quickly securing its place in the future of manufacturing.
\subsection{Internet of Things}
IoT devices offer a number of new opportunities for healthcare professionals to monitor patients, as well as for patients to monitor themselves. By extension, the variety of wearable IoT devices provide an array of benefits and challenges, for healthcare providers and their patients alike.Remote patient monitoring is the most common application of IoT devices for healthcare.IoT devices can help address these challenges by providing continuous, automatic monitoring of glucose levels in patients.a variety of small IoT devices are available for heart rate monitoring, freeing patients to move around as they like while ensuring that their hearts are monitored continuously. 
\subsection{Augmented Reality
}
Augmented Reality is yet another strong answer to what is disruptive technology in healthcare. It compliments reality with images and sounds and creates its own type of extended reality. The healthcare industry is using it mainly for its educational benefits.  Augmented reality’s potential offers vast opportunities for doctors to be able to work more efficiently.Allows doctors to record video and audio while examining a patient.Enables surgeons to get interactive augmented reality assistance during surgery.Assists in vein visualization and streams ultrasound.Allows for interactive printing and demos of how various drugs & devices affect the human body.Adds value to enhance doctors’ performance.Another benefit of augmented reality is it helps you show your patient how to exactly apply their medication or wash and dress their wound.


\end{document}